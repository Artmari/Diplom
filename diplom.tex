\documentclass{article}

    %\usepackage[cp1251]{inputenc} % указать кодировку 
    
    \usepackage[ukrainian]{babel} 
    \newtheorem{theorem}{Теорема}
    \usepackage{amssymb}
    \usepackage{amsmath}
    \usepackage[14pt]{extsizes}
    \textheight=24cm
    \textwidth=16cm
    \oddsidemargin=0pt
    \topmargin=-1.5cm
    \def\baselinestretch{1.5}
    \usepackage{amsfonts}
    \usepackage{graphicx}
    \usepackage{color}
    \usepackage{alltt}
    \parindent=1cm
    
\begin{document}

%==================ВСТУП===================================================================%
Стабілізація нелінійних систем є однією з найбільш цікавих та важливих задач теорії керування
Не дивлячись на великі досягнення останніх десятеліть ця задача досі не розв'язана в 
загальному випадку. Особливий інтерес викликають системи нестабілізовані за першим
наближенням. Один з найбільш універсальних методів стабілізації нелінійних систем є метод зворотнього ходу 
(backstepping в англомовній літературі). 

Метод зворотнього ходу - це рекурсивна процедура, в котрій поєднані задачі пошуку функції Ляпунова та відповідного закону
керування. Суть методу полягає у тому, що задача пошуку закону керування всієї системи розбиваеться на послідовність
відповідних підзадач для підсистем меншого порядку, для яких відомий закон керування та функція Ляпунова відомі.
Зі зростанням розмірності кожна додаткова фазова змінна входить як керування у цю підсистему. 
При такому підході є необхідним трикутний вигляд системи.
Трикутною системою називається система,
що в загальному випадку має вигляд: 
\begin{equation}
	\begin{cases}
        \dot x_1 = f_1(x_1, x_2)\\
        \dot x_2 = f_2(x_1, x_2 ,x_3)\\
        ...\\
        \dot x_{n-1} = f_{n-1}(x_1, ... ,x_n)\\
        \dot x_{n} = f_{n}(x_1, ... ,x_n,u)\\
	\end{cases}
\end{equation}
\pagebreak




%==================ЗАГАЛЬНИЙ ВИГЛЯД====================================================================%
Для того, щоб застосовувати метод бекстеппінгу система повинна мати наступний вигляд:
\begin{equation}
    \begin{cases}
        \dot x           = f_0(x)+g_0(x)\xi_1\\
        \dot \xi_1       = f_1(x, \xi_1)+g_{1}(x, x_1)\xi_2 \\
        \dot \xi_2       = f_2(x, \xi_1, \xi_2) + g_2(x, \xi_1, \xi_2)\xi_3 \\
        \dots\\
        \dot \xi_{n-1}   = f_{n-1}(x, \xi_1, \xi_2, ... ,\xi_{n-1}) 
        +g_{n-1}(x, \xi_1, \xi_2, ... ,\xi_{n-1}) \xi_n\\
        \dot \xi_{n}     = f_{n}(x, \xi_1, \xi_2, ... ,\xi_{n}) 
        +g_{n}(x, \xi_1, \xi_2, ... ,\xi_{n})u\\

	\end{cases}
\end{equation}

такі трикутні системи ще називають "strict-feedback" системи.\\
Де $x \in \mathbb{R}$, $n \leq 1$\\
$\xi_1, \xi_2, ... ,\xi_n \in \mathbb{R}$, $u \in \mathbb{R}$ - керування,\\ 
$f_i, g_i, i = 1 ... n $ -  відомі функції, $f_i(0,0, \dots, 0) = 0$, 
%========================================================================================================%
\pagebreak







%==================ПРИКЛАД_1===============================================================================%
Розглянемо систему:
\begin{equation}
	\begin{cases}
		\dot \xi_1 = f(\xi_1) + g(\xi_1)\xi_2 \\
		\dot\xi_2 = u
	\end{cases}
\end{equation}
Цю система може бути розглянута як дві підсистеми, а саме перша підсистема, де $\xi_2$ виступає як вхід, друга
підсистема, як інтегратор.Основна ідея побудови полягае у тому, щоб розгядати $\xi_2$ як (віртуальне) керування для
стабілізації $xi_1$. Вважаемо, що існує керування $\phi(\xi_1)$, таке, що нульова точка покою
системи $\dot \xi_1 = f(\xi_1) + g(\xi_1)\phi(\xi_1)$
асимптотично стійка.
Вважаемо, що для вибраного $\phi(\xi_1)$ функція Ляпунова $V(\xi_1)$  відома та задовільняє умові:

\begin{equation}
    \frac{\partial V(\xi)}{\partial \xi_1}(f(\xi_1)+g(\xi_1)\phi(\xi_1)) \leq
    -W(\xi_1), \forall \xi_1 \in \mathbb{R}
\end{equation}

До першого рівняння додамо та віднімемо $g(\xi_1)\phi(\xi_1)$

\begin{equation}
\dot\xi_1 = f(\xi_1)+g(\xi_1)\phi(\xi_1)-g(\xi_1)\phi(\xi_1)+g(\xi)\xi_2 =\\
f(\xi_1)+g(\xi_1)\phi(\xi_1)-g(\xi_1)(\phi(\xi_1)-\xi_2)
\end{equation}
Позначимо $e_{\xi_1} = \xi_2-\phi(\xi_1)$
Перепишемо систему в координатах $(\xi_1, e_{\xi_1})$

\begin{equation}
    \begin{cases}
    \dot \xi_1 = (f(\xi_1)+g(\xi_1)\phi(\xi_1))+g(\xi_1)e_{\xi_1}\\
    \dot e_{\xi_1} = u-\dot \phi(xi_1)\\  
    \end{cases}
\end{equation}
Обчислити $\dot\phi$ 

\begin{equation}
    \dot\phi = \frac{\partial \phi}{\partial \xi_1}(f(\xi_1)+g(\xi_1)\xi_2)
\end{equation}
Позначимо: $u = v + \dot\phi$, $v \in \mathbb{R}$.
Систему перепишемо:
\begin{equation}
    \begin{cases}
        \dot \xi_1 = (f(\xi_1)+g(\xi_1)\phi(\xi_1))+g(\xi_1)e_{\xi_1}\\
        \dot e_{\xi_1} = v\\
    \end{cases}
\end{equation}
Відмітемо, що система має асимтотично стійку нульову точку спокою $\xi_1$ коли $e_{\xi_1}$
Розглянемо функцію $V(\xi_1,\xi_2)$ - кандидат на функцію Ляпунова, що має вигляд:
\begin{equation}
    \dot V_2 = \frac{\partial V}{\partial \xi_1}(f(\xi_1)+g(\xi_1)\phi(\xi_1))+
    \frac{\partial V}{\partial \xi_1}e_{\xi_1}+e_{\xi_1}v \leq -W(\xi_1)+
    \frac{\partial V}{\partial \xi_1}e_{\xi_1}+e_{\xi_1}v
\end{equation}
У якості $\dot e_{\xi_1}$ беремо 
\begin{equation}
    v = - \frac{\partial V}{\partial \xi_1}g(\xi_1) - ke_{\xi_1}
\end{equation}

Параметр $k$ вибераємо додатним.
Отримаємо $V_2 \leq -W(\xi_1) - ke_{(\xi_1)^2}$ \\
Таким чином $\phi(0)=0$, $e_{\xi_1} -> 0$ нульова точка покою асимптотично стійка.
Кінцевий вигляд закону керування:

\begin{equation}
    u = \frac{\partial \phi}{\partial \xi_1}(f(\xi_1)+g(\xi_1)\xi_2)-
    \frac{\partial V}{\partial \xi_1}g(\xi_1)-k(\xi_2-\phi(\xi_1))
\end{equation}
%==========================================================================================================%
\pagebreak

%=======================SECTION_2=========================================================================%
Розглянемо наступну систему:
\begin{equation}
    \begin{cases}
    \dot x_1 = u \\
    \dot x_2 = x_1^3\\
    \dot x_3 = x_2^3\\
    \end{cases}
\end{equation}

Для стабілізації системи потрібно спочатку знайти функцію Ляпунова
та керування підсистеми, що на розмірність менше ніж система 
Для зручності перепишемо друге та третє рівняння системи
у координатах $(x_1,x_2)$:
\begin{equation}
    \begin{cases}
    \dot x_1 = x_1^3\\
    \dot x_2 = x_{1}^3\\
    \end{cases}
\end{equation}
До першого рівняння додамо та віднімемо вираз $(ax_1+bx_2)^3$:
\begin{equation}
    \begin{cases}
    \dot x_1 = x_1^3 - (ax_1+bx_2)^3 +(ax_1+bx_2)^3 \\
    \dot x_2 = x_{1}^3\\
    \end{cases}
\end{equation}
В якості допоміжного керування $u$ візмемо $u=ax_1+bx_2$, $a,b \in \mathbb{R}$
Для стабілізації підсистеми потрібно занйти такі  $a,b$ щоб система 
\begin{equation}
    \begin{cases}
    \dot x_1 = (ax_1+bx_2)^3\\
    \dot x_2 = x_{1}^3\\
    \end{cases}
\end{equation}
була стабілізованою.

До другого рівняння додамо та віднімемо 
\begin{equation}
    \begin{cases}
    \dot x_1 = (ax_1+bx_2)^3\\
    \dot x_2 = x_{1}^3 -\gamma^3(x_2)+\gamma^3(x_2)\\
    \end{cases}
\end{equation}
Де $\gamma(x_2) = \alpha x_2$
Функцію Ляпунова візьмемо у вигляді  
\begin{equation}
    V = \frac{1}{2}x_{2}^2 +\frac{1}{2}(x_1 - \gamma(x_2) )^2\\
\end{equation}

Знайдемо похідну функціїї Ляпунова в силу системи 
\begin{equation}
    \dot V = x_2(x_1^3 - \gamma^3(x_2)+\gamma^3(x_2)) + \\
    (x_1 - \gamma(x_2))(u^3-\frac{\partial \gamma}{\partial x_2}x_1^3)
\end{equation}

Після перетворень маємо
\begin{equation}
    \dot V = x_2\gamma^3(x_2)+(x_1 - \gamma(x_2))
    (x_2x_1^2 + x_2x_1\gamma(x_2) + x_{2}^3 + u^3- 
    \frac{\partial \gamma}{\partial x_2}x_{1}^3)
\end{equation}

Для того, щоб похідна функції Ляпунова $\dot V$ була від'ємною, потрібно
представити  $\dot V$ як квадратичну форму, що має вигляд $\dot V =(Gy,y)$ 
та вибрати матрицю $G$ таку, щоб  $\dot V =(Gy,y) < 0$.
Хочемо, щоб виконувалась наступна рівність:
\begin{equation}
    x_2x_1^2+x_2x_1\gamma(x_2) + x_2 \gamma^2(x_2) - 
    \frac{\partial x_2}{\partial x_2}x_1^3 + u^3 = 
    -p(x_1-\gamma(x_2))^3 +2\beta(x_1-\gamma(x_2))x_{2}^2 
\end{equation}

Візьмемо $\gamma(x_2)$ рівним $x_2\alpha$ де $\alpha \in \mathbb{R}$
Таким чином рівняння буде мати вигляд
\begin{equation}
x_2x_{1}^2 + \alpha_{2}^2x_1 + \alpha^2x_2^3-\alpha x_{1}^3 +
a^3x_{1}^3 + b^3x_{2}^3+ 
3ba^2x_{1}^2x_2+3ab^2x_1x_{2}^2 = 
-p(x_1^3 - \alpha^{3}x_{2}^3 - 3\alpha^{2}x_1x_{2}^2)+
2\beta x_1x_{2}^2 - 2\alpha \beta x_2^3
\end{equation}


\begin{equation}
    \dot V = \alpha^3x_{2}^4 - p(x_1-\gamma(x_2))^4
    +2\beta(x_1-\gamma(x_2))^2x_{2}^2 = 
    \alpha^3x_{2}^4 + 2\beta(x_1-\alpha x_2)^2x_{2}^2
    -p(x_1-\alpha x_2)^4 \leq 0
\end{equation}
Рівняння записуємо у вигляді квадратичної форми
$(Gy,y)$, де за рахунок вибору параметрів буде виконуватися
умова $\dot V =(Gy,y) < 0$ 
Таким чином підсистема з керуванням  $u^3=(ax_1+bx_2)^3$ 
є стабілізованою. 

\end{document}